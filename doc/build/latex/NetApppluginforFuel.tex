% Generated by Sphinx.
\def\sphinxdocclass{report}
\documentclass[letterpaper,10pt,english]{sphinxmanual}
\usepackage[utf8]{inputenc}
\DeclareUnicodeCharacter{00A0}{\nobreakspace}
\usepackage{cmap}
\usepackage[T1]{fontenc}

\usepackage{babel}
\usepackage{times}
\usepackage[Bjarne]{fncychap}
\usepackage{longtable}
\usepackage{sphinx}
\usepackage{multirow}
\usepackage{eqparbox}


\addto\captionsenglish{\renewcommand{\figurename}{Fig. }}
\addto\captionsenglish{\renewcommand{\tablename}{Table }}
\SetupFloatingEnvironment{literal-block}{name=Listing }



\title{NetApp plugin for Fuel Documentation}
\date{March 21, 2016}
\release{4.0-4.0.0-1}
\author{Mirantis Inc.}
\newcommand{\sphinxlogo}{}
\renewcommand{\releasename}{Release}
\setcounter{tocdepth}{1}
\makeindex

\makeatletter
\def\PYG@reset{\let\PYG@it=\relax \let\PYG@bf=\relax%
    \let\PYG@ul=\relax \let\PYG@tc=\relax%
    \let\PYG@bc=\relax \let\PYG@ff=\relax}
\def\PYG@tok#1{\csname PYG@tok@#1\endcsname}
\def\PYG@toks#1+{\ifx\relax#1\empty\else%
    \PYG@tok{#1}\expandafter\PYG@toks\fi}
\def\PYG@do#1{\PYG@bc{\PYG@tc{\PYG@ul{%
    \PYG@it{\PYG@bf{\PYG@ff{#1}}}}}}}
\def\PYG#1#2{\PYG@reset\PYG@toks#1+\relax+\PYG@do{#2}}

\expandafter\def\csname PYG@tok@gd\endcsname{\def\PYG@tc##1{\textcolor[rgb]{0.63,0.00,0.00}{##1}}}
\expandafter\def\csname PYG@tok@gu\endcsname{\let\PYG@bf=\textbf\def\PYG@tc##1{\textcolor[rgb]{0.50,0.00,0.50}{##1}}}
\expandafter\def\csname PYG@tok@gt\endcsname{\def\PYG@tc##1{\textcolor[rgb]{0.00,0.27,0.87}{##1}}}
\expandafter\def\csname PYG@tok@gs\endcsname{\let\PYG@bf=\textbf}
\expandafter\def\csname PYG@tok@gr\endcsname{\def\PYG@tc##1{\textcolor[rgb]{1.00,0.00,0.00}{##1}}}
\expandafter\def\csname PYG@tok@cm\endcsname{\let\PYG@it=\textit\def\PYG@tc##1{\textcolor[rgb]{0.25,0.50,0.56}{##1}}}
\expandafter\def\csname PYG@tok@vg\endcsname{\def\PYG@tc##1{\textcolor[rgb]{0.73,0.38,0.84}{##1}}}
\expandafter\def\csname PYG@tok@vi\endcsname{\def\PYG@tc##1{\textcolor[rgb]{0.73,0.38,0.84}{##1}}}
\expandafter\def\csname PYG@tok@mh\endcsname{\def\PYG@tc##1{\textcolor[rgb]{0.13,0.50,0.31}{##1}}}
\expandafter\def\csname PYG@tok@cs\endcsname{\def\PYG@tc##1{\textcolor[rgb]{0.25,0.50,0.56}{##1}}\def\PYG@bc##1{\setlength{\fboxsep}{0pt}\colorbox[rgb]{1.00,0.94,0.94}{\strut ##1}}}
\expandafter\def\csname PYG@tok@ge\endcsname{\let\PYG@it=\textit}
\expandafter\def\csname PYG@tok@vc\endcsname{\def\PYG@tc##1{\textcolor[rgb]{0.73,0.38,0.84}{##1}}}
\expandafter\def\csname PYG@tok@il\endcsname{\def\PYG@tc##1{\textcolor[rgb]{0.13,0.50,0.31}{##1}}}
\expandafter\def\csname PYG@tok@go\endcsname{\def\PYG@tc##1{\textcolor[rgb]{0.20,0.20,0.20}{##1}}}
\expandafter\def\csname PYG@tok@cp\endcsname{\def\PYG@tc##1{\textcolor[rgb]{0.00,0.44,0.13}{##1}}}
\expandafter\def\csname PYG@tok@gi\endcsname{\def\PYG@tc##1{\textcolor[rgb]{0.00,0.63,0.00}{##1}}}
\expandafter\def\csname PYG@tok@gh\endcsname{\let\PYG@bf=\textbf\def\PYG@tc##1{\textcolor[rgb]{0.00,0.00,0.50}{##1}}}
\expandafter\def\csname PYG@tok@ni\endcsname{\let\PYG@bf=\textbf\def\PYG@tc##1{\textcolor[rgb]{0.84,0.33,0.22}{##1}}}
\expandafter\def\csname PYG@tok@nl\endcsname{\let\PYG@bf=\textbf\def\PYG@tc##1{\textcolor[rgb]{0.00,0.13,0.44}{##1}}}
\expandafter\def\csname PYG@tok@nn\endcsname{\let\PYG@bf=\textbf\def\PYG@tc##1{\textcolor[rgb]{0.05,0.52,0.71}{##1}}}
\expandafter\def\csname PYG@tok@no\endcsname{\def\PYG@tc##1{\textcolor[rgb]{0.38,0.68,0.84}{##1}}}
\expandafter\def\csname PYG@tok@na\endcsname{\def\PYG@tc##1{\textcolor[rgb]{0.25,0.44,0.63}{##1}}}
\expandafter\def\csname PYG@tok@nb\endcsname{\def\PYG@tc##1{\textcolor[rgb]{0.00,0.44,0.13}{##1}}}
\expandafter\def\csname PYG@tok@nc\endcsname{\let\PYG@bf=\textbf\def\PYG@tc##1{\textcolor[rgb]{0.05,0.52,0.71}{##1}}}
\expandafter\def\csname PYG@tok@nd\endcsname{\let\PYG@bf=\textbf\def\PYG@tc##1{\textcolor[rgb]{0.33,0.33,0.33}{##1}}}
\expandafter\def\csname PYG@tok@ne\endcsname{\def\PYG@tc##1{\textcolor[rgb]{0.00,0.44,0.13}{##1}}}
\expandafter\def\csname PYG@tok@nf\endcsname{\def\PYG@tc##1{\textcolor[rgb]{0.02,0.16,0.49}{##1}}}
\expandafter\def\csname PYG@tok@si\endcsname{\let\PYG@it=\textit\def\PYG@tc##1{\textcolor[rgb]{0.44,0.63,0.82}{##1}}}
\expandafter\def\csname PYG@tok@s2\endcsname{\def\PYG@tc##1{\textcolor[rgb]{0.25,0.44,0.63}{##1}}}
\expandafter\def\csname PYG@tok@nt\endcsname{\let\PYG@bf=\textbf\def\PYG@tc##1{\textcolor[rgb]{0.02,0.16,0.45}{##1}}}
\expandafter\def\csname PYG@tok@nv\endcsname{\def\PYG@tc##1{\textcolor[rgb]{0.73,0.38,0.84}{##1}}}
\expandafter\def\csname PYG@tok@s1\endcsname{\def\PYG@tc##1{\textcolor[rgb]{0.25,0.44,0.63}{##1}}}
\expandafter\def\csname PYG@tok@ch\endcsname{\let\PYG@it=\textit\def\PYG@tc##1{\textcolor[rgb]{0.25,0.50,0.56}{##1}}}
\expandafter\def\csname PYG@tok@m\endcsname{\def\PYG@tc##1{\textcolor[rgb]{0.13,0.50,0.31}{##1}}}
\expandafter\def\csname PYG@tok@gp\endcsname{\let\PYG@bf=\textbf\def\PYG@tc##1{\textcolor[rgb]{0.78,0.36,0.04}{##1}}}
\expandafter\def\csname PYG@tok@sh\endcsname{\def\PYG@tc##1{\textcolor[rgb]{0.25,0.44,0.63}{##1}}}
\expandafter\def\csname PYG@tok@ow\endcsname{\let\PYG@bf=\textbf\def\PYG@tc##1{\textcolor[rgb]{0.00,0.44,0.13}{##1}}}
\expandafter\def\csname PYG@tok@sx\endcsname{\def\PYG@tc##1{\textcolor[rgb]{0.78,0.36,0.04}{##1}}}
\expandafter\def\csname PYG@tok@bp\endcsname{\def\PYG@tc##1{\textcolor[rgb]{0.00,0.44,0.13}{##1}}}
\expandafter\def\csname PYG@tok@c1\endcsname{\let\PYG@it=\textit\def\PYG@tc##1{\textcolor[rgb]{0.25,0.50,0.56}{##1}}}
\expandafter\def\csname PYG@tok@o\endcsname{\def\PYG@tc##1{\textcolor[rgb]{0.40,0.40,0.40}{##1}}}
\expandafter\def\csname PYG@tok@kc\endcsname{\let\PYG@bf=\textbf\def\PYG@tc##1{\textcolor[rgb]{0.00,0.44,0.13}{##1}}}
\expandafter\def\csname PYG@tok@c\endcsname{\let\PYG@it=\textit\def\PYG@tc##1{\textcolor[rgb]{0.25,0.50,0.56}{##1}}}
\expandafter\def\csname PYG@tok@mf\endcsname{\def\PYG@tc##1{\textcolor[rgb]{0.13,0.50,0.31}{##1}}}
\expandafter\def\csname PYG@tok@err\endcsname{\def\PYG@bc##1{\setlength{\fboxsep}{0pt}\fcolorbox[rgb]{1.00,0.00,0.00}{1,1,1}{\strut ##1}}}
\expandafter\def\csname PYG@tok@mb\endcsname{\def\PYG@tc##1{\textcolor[rgb]{0.13,0.50,0.31}{##1}}}
\expandafter\def\csname PYG@tok@ss\endcsname{\def\PYG@tc##1{\textcolor[rgb]{0.32,0.47,0.09}{##1}}}
\expandafter\def\csname PYG@tok@sr\endcsname{\def\PYG@tc##1{\textcolor[rgb]{0.14,0.33,0.53}{##1}}}
\expandafter\def\csname PYG@tok@mo\endcsname{\def\PYG@tc##1{\textcolor[rgb]{0.13,0.50,0.31}{##1}}}
\expandafter\def\csname PYG@tok@kd\endcsname{\let\PYG@bf=\textbf\def\PYG@tc##1{\textcolor[rgb]{0.00,0.44,0.13}{##1}}}
\expandafter\def\csname PYG@tok@mi\endcsname{\def\PYG@tc##1{\textcolor[rgb]{0.13,0.50,0.31}{##1}}}
\expandafter\def\csname PYG@tok@kn\endcsname{\let\PYG@bf=\textbf\def\PYG@tc##1{\textcolor[rgb]{0.00,0.44,0.13}{##1}}}
\expandafter\def\csname PYG@tok@cpf\endcsname{\let\PYG@it=\textit\def\PYG@tc##1{\textcolor[rgb]{0.25,0.50,0.56}{##1}}}
\expandafter\def\csname PYG@tok@kr\endcsname{\let\PYG@bf=\textbf\def\PYG@tc##1{\textcolor[rgb]{0.00,0.44,0.13}{##1}}}
\expandafter\def\csname PYG@tok@s\endcsname{\def\PYG@tc##1{\textcolor[rgb]{0.25,0.44,0.63}{##1}}}
\expandafter\def\csname PYG@tok@kp\endcsname{\def\PYG@tc##1{\textcolor[rgb]{0.00,0.44,0.13}{##1}}}
\expandafter\def\csname PYG@tok@w\endcsname{\def\PYG@tc##1{\textcolor[rgb]{0.73,0.73,0.73}{##1}}}
\expandafter\def\csname PYG@tok@kt\endcsname{\def\PYG@tc##1{\textcolor[rgb]{0.56,0.13,0.00}{##1}}}
\expandafter\def\csname PYG@tok@sc\endcsname{\def\PYG@tc##1{\textcolor[rgb]{0.25,0.44,0.63}{##1}}}
\expandafter\def\csname PYG@tok@sb\endcsname{\def\PYG@tc##1{\textcolor[rgb]{0.25,0.44,0.63}{##1}}}
\expandafter\def\csname PYG@tok@k\endcsname{\let\PYG@bf=\textbf\def\PYG@tc##1{\textcolor[rgb]{0.00,0.44,0.13}{##1}}}
\expandafter\def\csname PYG@tok@se\endcsname{\let\PYG@bf=\textbf\def\PYG@tc##1{\textcolor[rgb]{0.25,0.44,0.63}{##1}}}
\expandafter\def\csname PYG@tok@sd\endcsname{\let\PYG@it=\textit\def\PYG@tc##1{\textcolor[rgb]{0.25,0.44,0.63}{##1}}}

\def\PYGZbs{\char`\\}
\def\PYGZus{\char`\_}
\def\PYGZob{\char`\{}
\def\PYGZcb{\char`\}}
\def\PYGZca{\char`\^}
\def\PYGZam{\char`\&}
\def\PYGZlt{\char`\<}
\def\PYGZgt{\char`\>}
\def\PYGZsh{\char`\#}
\def\PYGZpc{\char`\%}
\def\PYGZdl{\char`\$}
\def\PYGZhy{\char`\-}
\def\PYGZsq{\char`\'}
\def\PYGZdq{\char`\"}
\def\PYGZti{\char`\~}
% for compatibility with earlier versions
\def\PYGZat{@}
\def\PYGZlb{[}
\def\PYGZrb{]}
\makeatother

\renewcommand\PYGZsq{\textquotesingle}

\begin{document}

\maketitle
\tableofcontents
\phantomsection\label{index::doc}


Contents:


\chapter{NetApp plugin}
\label{description:netapp-plugin}\label{description::doc}\label{description:guide-to-the-cinder-netapp-plugin-ver-4-0-0}
NetApp plugin provides support of ONTAP and E-series storage clusters to Cinder.
NetApp plugin uses NetApp unified driver; the latter is a
block storage driver that supports multiple storage families and protocols.
A storage family corresponds to storage systems built on different NetApp technologies
such as clustered Data ONTAP, Data ONTAP operating in 7-Mode,
and E-Series.
The storage protocol refers to the protocol used to initiate data
storage and access operations on those storage systems like iSCSI and NFS.
The NetApp unified driver can be configured to provision and manage OpenStack volumes
on the given storage family using the specified storage protocol.
The OpenStack volumes can then be used for accessing and storing data with
the storage protocol on the storage family system.
The NetApp unified driver is an extensible interface that can support new
storage families and protocols.


\section{Requirements}
\label{description:requirements}
\begin{tabulary}{\linewidth}{|L|L|}
\hline
\textsf{\relax 
Requirement
} & \textsf{\relax 
Version/Comment
}\\
\hline
Fuel
 & 
8.0
\\
\hline
ONTAP or E-Series
 & 
All storage family is supported.
\\
\hline\end{tabulary}



\section{Prerequisites}
\label{description:prerequisites}\begin{itemize}
\item {} 
If you plan to use the plugin with \textbf{ONTAP}, please make sure that it
is configured, up and running. For instructions, see \href{http://mysupport.netapp.com/documentation/productlibrary/index.html?productID=30092}{the official
NetApp ONTAP documentation}.

\item {} 
If you plan to use the plugin with \textbf{E-Series}, please make sure that it
is configured, up and running. For instructions, see \href{https://mysupport.netapp.com/info/web/ECMP1658252.html}{the official
NetApp E-Series documentation}.

\end{itemize}


\chapter{Installing NetApp plugin}
\label{installation:installing-netapp-plugin}\label{installation::doc}
To install the Cinder Netapp plugin, follow these steps:
\begin{enumerate}
\item {} 
Download it from the \href{https://www.mirantis.com/products/openstack-drivers-and-plugins/fuel-plugins/}{Fuel Plugins Catalog}.

\item {} 
Copy the plugin's RPM to the Fuel Master node (if you don't
have the Fuel Master node, please see \href{https://docs.mirantis.com/openstack/fuel/fuel-8.0/quickstart-guide.html\#installing-mirantis-openstack-manually}{the official
Mirantis OpenStack documentation}):

\begin{Verbatim}[commandchars=\\\{\}]
[root@home \PYGZti{}]\PYGZsh{} scp cinder\PYGZus{}netapp\PYGZhy{}4.0\PYGZhy{}4.0.0\PYGZhy{}1.noarch.rpm root@fuel\PYGZhy{}master:/tmp
\end{Verbatim}

\item {} 
Log into Fuel Master node and install the plugin using the
\href{https://docs.mirantis.com/openstack/fuel/fuel-8.0/user-guide.html\#using-fuel-cli}{Fuel CLI}:

\begin{Verbatim}[commandchars=\\\{\}]
[root@fuel\PYGZhy{}master \PYGZti{}]\PYGZsh{} fuel plugins \PYGZhy{}\PYGZhy{}install cinder\PYGZus{}netapp\PYGZhy{}4.0\PYGZhy{}4.0.0\PYGZhy{}1.noarch.rpm
\end{Verbatim}

\item {} 
Verify that the plugin is installed correctly:

\begin{Verbatim}[commandchars=\\\{\}]
[root@fuel\PYGZhy{}master \PYGZti{}]\PYGZsh{} fuel plugins
id \textbar{} name          \textbar{} version \textbar{} package\PYGZus{}version
\PYGZhy{}\PYGZhy{}\PYGZhy{}\textbar{}\PYGZhy{}\PYGZhy{}\PYGZhy{}\PYGZhy{}\PYGZhy{}\PYGZhy{}\PYGZhy{}\PYGZhy{}\PYGZhy{}\PYGZhy{}\PYGZhy{}\PYGZhy{}\PYGZhy{}\PYGZhy{}\PYGZhy{}\textbar{}\PYGZhy{}\PYGZhy{}\PYGZhy{}\PYGZhy{}\PYGZhy{}\PYGZhy{}\PYGZhy{}\PYGZhy{}\PYGZhy{}\textbar{}\PYGZhy{}\PYGZhy{}\PYGZhy{}\PYGZhy{}\PYGZhy{}\PYGZhy{}\PYGZhy{}\PYGZhy{}\PYGZhy{}\PYGZhy{}\PYGZhy{}\PYGZhy{}\PYGZhy{}\PYGZhy{}\PYGZhy{}\PYGZhy{}
1  \textbar{} cinder\PYGZus{}netapp \textbar{} 4.0.0   \textbar{} 4.0.0
\end{Verbatim}

\end{enumerate}

Once the Fuel Cinder NetApp  plugin has been installed, you can
create OpenStack environments that use NetApp storage as a Cinder backend.


\chapter{Configuring NetApp plugin}
\label{guide::doc}\label{guide:fuel-plugins-catalog}\label{guide:configuring-netapp-plugin}\begin{enumerate}
\item {} 
Create an OpenStack environment using the Fuel UI wizard:

\includegraphics[width=0.900\linewidth]{{create_env}.png}

\item {} 
Finish environment creation following
\href{https://docs.mirantis.com/openstack/fuel/fuel-8.0/user-guide.html\#create-a-new-openstack-environment}{the instructions}.

\item {} 
Once the environment is created, open the \textbf{Settings} tab of the Fuel Web UI
and then \textbf{Storage}. Scroll down the page. Select the \textbf{Cinder and NetApp integration}
checkbox:

\includegraphics[width=0.400\linewidth]{{select-checkbox}.png}

\item {} 
Configure the plugin.Select \textbf{Multibackend enabled} checkbox
if you would like NetApp driver to be used as the Cinder Multibackend feature:

\includegraphics[width=0.500\linewidth]{{multibackend}.png}

\item {} 
Choose storage family and storage protocol. Several options are available.
\begin{itemize}
\item {} 
If you plan to use ONTAP cluster mode through NFS, click \textbf{Ontap Cluster}
radiobutton and select \emph{nfs} option in \textbf{Netapp storage protocol}.
You should also choose NetApp transport type (http or https).
Specify the following parameters in the text fields:
\begin{itemize}
\item {} 
Netapp username

\item {} 
Netapp password

\item {} 
Netapp server hostname

\item {} 
NFS server

\item {} 
NFS share(s)

\item {} 
Netapp Vserver

\end{itemize}

\includegraphics[width=1.000\linewidth]{{cmode_nfs}.png}

\item {} 
If you plan to use ONTAP cluster mode through iSCSI, click \textbf{Ontap Cluster}
radiobutton and select \emph{iscsi} option in \textbf{Netapp storage protocol}.
You should also choose NetApp transport type (http or https).
Specify the following parameters in the text fields:
\begin{itemize}
\item {} 
Netapp username

\item {} 
Netapp password

\item {} 
Netapp server hostname

\item {} 
Netapp Vserver

\end{itemize}

\includegraphics[width=1.000\linewidth]{{cmode_iscsi}.png}

\item {} 
If you plan to use ONTAP 7 mode through NFS, click \textbf{Ontap 7mode}
radiobutton and select \emph{nfs} option in \textbf{Netapp storage protocol}.
You should also choose NetApp transport type (http or https).
Specify the following parameters in the text fields:
\begin{itemize}
\item {} 
Netapp username

\item {} 
Netapp password

\item {} 
Netapp server hostname

\item {} 
NFS server

\item {} 
NFS share(s)

\end{itemize}

\includegraphics[width=1.000\linewidth]{{7mode_nfs}.png}

\item {} 
If you plan to use ONTAP 7 mode through iSCSI, click \textbf{Ontap 7mode}
radiobutton and select \emph{iscsi} option in \textbf{Netapp storage protocol}.
You should also choose NetApp transport type (http or https).
Specify the following parameters in the text fields:

\end{itemize}
\begin{quote}
\begin{itemize}
\item {} 
Netapp username

\item {} 
Netapp password

\item {} 
Netapp server hostname

\end{itemize}

\includegraphics[width=1.000\linewidth]{{7mode_iscsi}.png}
\end{quote}
\begin{itemize}
\item {} 
If you plan to use E-series, click \textbf{E-Series}
radiobutton and select the only available \emph{iscsi} option in \textbf{Netapp storage protocol}.
You should also choose NetApp transport type (http or https).
Specify the following parameters in the text fields: please specify the following parameters:
\begin{itemize}
\item {} 
Netapp username

\item {} 
Netapp password

\item {} 
Netapp server hostname

\item {} 
Netapp controller IPs

\item {} 
Netapp SA password

\item {} 
Storage pools

\end{itemize}

\includegraphics[width=1.000\linewidth]{{eseries}.png}

\end{itemize}

\item {} 
Using \emph{Nodes} tab,
\href{https://docs.mirantis.com/openstack/fuel/fuel-8.0/user-guide.html\#add-nodes-to-the-environment}{add nodes and assign roles to them}.
Please, note that all controller nodes should be configured with Cinder role.

\item {} 
Press \href{https://docs.mirantis.com/openstack/fuel/fuel-8.0/user-guide.html\#deploy-changes}{Deploy button}
once you are done with environment configuration.

\item {} 
When the deployment is done, you may perform functional testing.
You can find instructions in \href{http://content.mirantis.com/Mirantis-NetApp-Reference-Architecture-Landing-Page.html}{NetApp Mirantis Unlocked Reference Architecture}, paragraph 8.3.

\end{enumerate}


\chapter{Licenses}
\label{licenses:licenses}\label{licenses::doc}
The plugin does not include any third-party
components and is published under Apache 2.0 license.


\chapter{Appendix}
\label{appendix:appendix}\label{appendix::doc}\begin{enumerate}
\item {} 
\href{http://mysupport.netapp.com/documentation/productlibrary/index.html?productID=30092}{ONTAP documentation}

\item {} 
\href{https://mysupport.netapp.com/info/web/ECMP1658252.html}{E-Series documentation}

\end{enumerate}



\renewcommand{\indexname}{Index}
\printindex
\end{document}
